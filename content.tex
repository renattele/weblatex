% ==== ГОСТ 34: ОСНОВНОЙ КОНТЕНТ ====
\documentclass[14pt,a4paper]{extarticle}
% ==== ГОСТ 34: ПРЕАМБУЛА И МАКРОСЫ (без \documentclass и без \begin{document}) ====

% Кодировки и язык (pdfLaTeX; для XeLaTeX используйте fontspec/polyglossia)
\usepackage[T2A]{fontenc}
\usepackage[utf8]{inputenc}
\usepackage[russian]{babel}
\hyphenation{долж-ность ис-пол-ни-те-ля под-пись функ-ци-о-наль-ные ста-ди-я ком-мен-та-рий про-гра-мма ме-то-ди-ка ис-пы-та-ний}
\uchyph=1 % разрешаем переносы слов с прописными буквами

% Поля, абзацы, межстрочный
\usepackage[left=30mm,right=10mm,top=20mm,bottom=20mm]{geometry}
\usepackage{indentfirst}
\usepackage{setspace}
\onehalfspacing
\setlength{\parindent}{1.25cm}
\setlength{\parskip}{0pt}

% Таблицы, рисунки, подписи
\usepackage{graphicx}
\usepackage{array}
\usepackage{ragged2e} % гибкие выравнивания с переносами в таблицах
\usepackage{longtable}
\usepackage{caption}
\DeclareCaptionLabelSeparator{emdash}{\textemdash\ }
\captionsetup{labelsep=emdash, justification=raggedright, singlelinecheck=false}
\renewcommand{\figurename}{Рисунок}
\renewcommand{\tablename}{Таблица}

% Глобальный стиль таблиц: рамка/сетка, заметные линии
\setlength{\arrayrulewidth}{0.8pt}
\setlength{\tabcolsep}{6pt}
\renewcommand{\arraystretch}{1.2}

% Обёртки-окружения для таблиц в стиле ГОСТ
\newenvironment{GOSTtabular}[1]{%
  \begin{tabular}{|#1|}\hline
}{%
  \end{tabular}
}
\newenvironment{GOSTlongtable}[1]{%
  \setlength{\LTleft}{0pt}\setlength{\LTright}{0pt}% выравнивание по центру
  \centering
  \begin{longtable}{|#1|}
  \hline
}{%
  \end{longtable}
}

% Полезные p-колонки
\newcolumntype{L}[1]{>{\RaggedRight\arraybackslash}p{\dimexpr#1-2\tabcolsep\relax}}
\newcolumntype{C}[1]{>{\Centering\arraybackslash}p{\dimexpr#1-2\tabcolsep\relax}}
\newcolumntype{R}[1]{>{\RaggedLeft\arraybackslash}p{\dimexpr#1-2\tabcolsep\relax}}

% Списки и глубина нумерации (до 1.1.1.1)
\usepackage{enumitem}
\setlist{nosep,topsep=0pt,partopsep=0pt}
\setcounter{secnumdepth}{4}
\usepackage{titlesec}
\titleformat{\section}{\normalfont\bfseries\large}{\thesection}{1em}{}
\titleformat{\subsection}{\normalfont\bfseries\normalsize}{\thesubsection}{1em}{}
\titleformat{\subsubsection}{\normalfont\bfseries\normalsize}{\thesubsubsection}{1em}{}
\titleformat{\paragraph}{\normalfont\bfseries\normalsize}{\theparagraph}{1em}{}

% Ссылки (ч/б без цветов)
\usepackage[unicode]{hyperref}
\hypersetup{colorlinks=true, linkcolor=black, urlcolor=black, citecolor=black}

% ===== Макросы метаданных титула
\newcommand{\OrgName}{ООО «Aivise Technologies»}
\newcommand{\SystemName}{Автоматизированная образовательная система персонализированного сопровождения «Aivise»}
\newcommand{\DocType}{ТЕХНИЧЕСКОЕ ЗАДАНИЕ} % / ЧАСТНОЕ ТЕХНИЧЕСКОЕ ЗАДАНИЕ
\newcommand{\TopicCode}{AIVISE-EDU/2025-ТЗ}
\newcommand{\City}{Казань}
\newcommand{\Year}{2025}

% ===== Титульный лист (вызывается один раз в основном файле)
\newcommand{\MakeTitlePage}{%
\begin{titlepage}
\thispagestyle{empty}
\begin{center}
{\large \OrgName\par}
\vspace{25mm}
{\Large \bfseries \SystemName\par}
\vspace{10mm}
{\LARGE \bfseries \DocType\par}
\vspace{5mm}
{\large Шифр темы: \TopicCode\par}
\vfill
{\City\par}
{\Year\par}
\end{center}
\end{titlepage}
}

% ===== Перечень сокращений — таблица (GOSTlongtable)
\newcommand{\AbbrevTable}{%
\section*{ПЕРЕЧЕНЬ СОКРАЩЕНИЙ И УСЛОВНЫХ НАИМЕНОВАНИЙ}
\addcontentsline{toc}{section}{ПЕРЕЧЕНЬ СОКРАЩЕНИЙ И УСЛОВНЫХ НАИМЕНОВАНИЙ}
\begin{GOSTlongtable}{L{0.30\textwidth}|L{0.62\textwidth}}
\textbf{Термин, сокращение} & \textbf{Определение} \\ \hline
АС & Автоматизированная система \\ \hline
АСУ & Автоматизированная система управления \\ \hline
САПР & Система автоматизированного проектирования \\ \hline
КТС & Комплекс технических средств \\ \hline
% Добавляйте строки ниже:
% АСНИ & Автоматизированная система научных исследований \\ \hline
\end{GOSTlongtable}
\clearpage
}

% ===== Пример обычной таблицы (унифицированный стиль ГОСТ)
\newcommand{\DemoTable}{%
\begin{table}[h]
\centering
\caption{Наименование таблицы}
\begin{GOSTtabular}{L{0.3\textwidth}|L{0.65\textwidth}}
\textbf{Заголовок столбца 1} & \textbf{Заголовок столбца 2} \\ \hline
Ячейка 1 & Ячейка 2 \\ \hline
Ячейка 3 & Ячейка 4 \\ \hline
\end{GOSTtabular}
\end{table}
}

% ===== Таблицы "СОСТАВИЛИ / СОГЛАСОВАНО" в стиле эталона
\newcommand{\SignatureTables}{%
  \clearpage
  \thispagestyle{plain}%
  \begin{center}
    {\small \TopicCode. Техническое задание / Частное техническое задание}
  \end{center}

  \vspace{2mm}
  \begin{center}\textbf{\MakeUppercase{СОСТАВИЛИ}}\end{center}
  \vspace{1mm}

  \begin{center}
  \begin{minipage}{\textwidth}
    \setlength{\tabcolsep}{3pt}% ужимаем отступы внутри таблицы подписей
    \begin{GOSTtabular}{L{45mm}|L{32mm}|L{37mm}|C{24mm}|C{18mm}}
      \textbf{Наименование организации, предприятия} &
      \textbf{Должность исполнителя} &
      \textbf{Фамилия, имя, отчество} &
      \textbf{Подпись} &
      \textbf{Дата} \\ \hline
      & & & & \\ \hline
      & & & & \\ \hline
      & & & & \\ \hline
      & & & & \\ \hline
    \end{GOSTtabular}
  \end{minipage}
  \end{center}

  \vspace{6mm}
  \begin{center}\textbf{\MakeUppercase{СОГЛАСОВАНО}}\end{center}
  \vspace{1mm}

  \begin{center}
  \begin{minipage}{\textwidth}
    \setlength{\tabcolsep}{3pt}% ужимаем отступы внутри таблицы подписей
    \begin{GOSTtabular}{L{45mm}|L{32mm}|L{37mm}|C{24mm}|C{18mm}}
      \textbf{Наименование организации, предприятия} &
      \textbf{Должность исполнителя} &
      \textbf{Фамилия, имя, отчество} &
      \textbf{Подпись} &
      \textbf{Дата} \\ \hline
      & & & & \\ \hline
      & & & & \\ \hline
      & & & & \\ \hline
      & & & & \\ \hline
    \end{GOSTtabular}
  \end{minipage}
  \end{center}

  \clearpage
}


\begin{document}

% ---- Титул (ровно один раз)
\MakeTitlePage

% ---- Перечень сокращений (таблица)
\AbbrevTable

% ---- Оглавление
\tableofcontents
\clearpage

% ===== 1 Общие сведения
\section{Общие сведения}
\subsection{Полное наименование системы и её условное обозначение}
Автоматизированная образовательная система персонализированного сопровождения «Aivise» (далее — Система).
Условное обозначение: АС Aivise.

\subsection{Шифр темы или шифр (номер) договора}
Рабочий шифр темы: AIVISE-EDU/2025-ТЗ. Окончательный регистрационный номер договора и шифр проекта фиксируются в договоре на разработку прототипа и в приказе Заказчика о запуске пилотной программы.

\subsection{Наименование предприятий (объединений) разработчика и заказчика (пользователя) системы и их реквизиты}
Создание Системы инициировано консорциумом общеобразовательных организаций и стартапом, специализирующимся на AI-наставниках для школ.

\noindent\textbf{Заказчик:} Консорциум пилотных общеобразовательных школ «Aivise пилот». Юридический адрес и банковские реквизиты уточняются на этапе заключения договора; контактное лицо — заместитель директора по цифровой трансформации пилотной школы.\\
\textbf{Исполнитель:} ООО «Aivise Technologies». Юридический адрес: 999999, г. Казань, ул. Придуманная, д. 0, офис 404. ИНН и банковские реквизиты фиксируются в договоре.\\
\textbf{Соисполнители:} АНО «Центр педагогических инноваций» (методическое сопровождение) и консультанты по цифровой трансформации школ.\\[0.5\baselineskip]
Реквизиты сторон приводятся в договоре и приложениях к нему; актуализация осуществляется при переходе к пилотным внедрениям.

\subsection{Перечень документов, на основании которых создаётся система, кем и когда утверждены эти документы}
При разработке ТЗ используются следующие документы и материалы:
\begin{enumerate}
  \item «Концепция платформы Aivise» (версия 1.1 от 20.09.2025), утверждённая рабочей группой Заказчика.
  \item Отчёт маркетингового исследования «Практики внедрения AI в школах РФ» (сентябрь 2025 года), подготовленный ООО «Aivise Technologies».
  \item Протоколы интервью с представителями учителей, родителей и администрации (октябрь 2025 года).
  \item Презентационные материалы стратегии вывода продукта на рынок, согласованные с инвесторами (ноябрь 2025 года).
\end{enumerate}

\subsection{Плановые сроки начала и окончания работы по созданию системы}
Начало работ: январь 2025 года (стадия концепции завершена).
\begin{enumerate}
  \item \textbf{0–6 месяцев} — разработка функционального прототипа платформы, сбор учебных кейсов, ограниченное тестирование на фокус-группе учителей и учеников.
  \item \textbf{6–12 месяцев} — запуск пилотных проектов в нескольких школах, итеративное совершенствование AI-наставников и интерфейсов, подготовка эксплуатационной документации.
  \item \textbf{12–24 месяцев} — масштабирование пилота до сети школ, подготовка к коммерческому запуску, создание служб поддержки и интеграция с ИТ-инфраструктурой школ.
\end{enumerate}
Завершение стадии разработки и подготовки к промышленной эксплуатации планируется на четвёртый квартал 2025 года.

\subsection{Сведения об источниках и порядке финансирования работ}
Финансирование проекта планируется поэтапно из средств сид-инвестиций и грантовых программ цифровой трансформации образования.
\begin{itemize}
  \item \textbf{Первый год (MVP и пилот)} — ориентировочный бюджет 0,8 млн USD. Основные статьи: разработка платформы (около 0,48 млн USD, включая ML-инженеров, backend, frontend, продакт-менеджера, дизайнера и QA), создание учебного контента (около 0,10 млн USD), маркетинг и продвижение пилота (около 0,08 млн USD), инфраструктура и операционные расходы (около 0,13 млн USD).
  \item \textbf{Расширение во второй год} — около 1,6 млн USD на масштабирование команды, рост серверных мощностей, интеграции со школьными системами и активное продвижение.
\end{itemize}
Поступление средств согласуется с дорожной картой: транш на разработку прототипа перечисляется в начале проекта, транш на пилот — по факту готовности демонстрационного стенда, транш на масштабирование — по результатам приёмки пилотных внедрений.

\subsection{Порядок оформления и предъявления заказчику результатов работ}
Результаты каждого этапа оформляются комплектами документации по ГОСТ 34.201 с электронными копиями в системе управления проектами Заказчика. Прототип и отчёт об опытной эксплуатации передаются комиссии Заказчика не позднее 10 рабочих дней после завершения соответствующего этапа. Акты приёма-передачи подписываются руководителем проекта Исполнителя и уполномоченным представителем Заказчика.

\subsection{Уточнение и дополнение ТЗ на АС}
Изменения требований и функциональных характеристик оформляются дополнениями к настоящему ТЗ. Дополнения согласуются техническим комитетом проекта и утверждаются в порядке, установленном для документации по ГОСТ 34. Приоритетными являются изменения, вытекающие из результатов пилотных внедрений и обратной связи от учителей и родителей.

% ===== 2 Назначение и цели создания (развития) системы
\section{Назначение и цели создания (развития) системы}
\subsection{Назначение системы}
АС Aivise предназначена для поддержки образовательного процесса в общеобразовательных школах путём предоставления персонализированных AI-наставников, автоматизации рутинных задач учителя и обеспечения прозрачного мониторинга прогресса учащихся. Система внедряется в образовательных организациях, где требуется масштабировать качественную обратную связь и персональные траектории обучения.

Система обслуживает ключевые категории пользователей:
\begin{itemize}
  \item учителя — получают автоматизированную проверку домашних заданий, аналитику по классу и рекомендации по дополнительным материалам;
  \item ученики — взаимодействуют с персональным наставником, который объясняет тему столько раз, сколько требуется, и формирует адаптивные задания;
  \item родители — имеют прозрачный доступ к прогрессу ребёнка и уведомления о критических событиях;
  \item администрация школы — отслеживает эффективность образовательного процесса и загрузку педагога, принимает решения по масштабированию практик.
\end{itemize}

Конкурентное позиционирование Системы основано на специализации под школьный контекст. В отличие от сервисов типа Kahoot, Aivise предоставляет не только элементы геймификации, но и полноценного цифрового наставника. По сравнению с универсальными решениями на основе GPT, Система поставляется с тысячами преднастроенных ботов по предметам и не требует от педагога подготовки промптов. Электронные журналы рассматриваются как смежные продукты, решающие учётные функции; Aivise фокусируется на обучении и взаимодействии с учениками.

\subsection{Цели создания системы}
Целью разработки является создание масштабируемой AI-платформы, обеспечивающей индивидуальный подход к каждому ученику при сохранении управляемой нагрузки на учителя. Ключевые целевые ориентиры:
\begin{enumerate}
  \item Обеспечить персонализацию обучения: формирование индивидуальных дорожных карт и адаптивных заданий для каждого ученика с учётом предметной сложности и темпа освоения.
  \item Разгрузить учителя от рутинных операций: автоматизировать проверку типовых заданий, сбор статистики, генерацию рекомендаций и уведомлений.
  \item Обеспечить бесшовную интеграцию в существующие школьные процессы: использовать привычные каналы коммуникации, минимизировать бюрократическую нагрузку, поддерживать работу через веб-интерфейс и интеграции с электронными журналами.
\end{enumerate}
Показателями достижения целей являются доля учащихся с завершёнными индивидуальными траекториями, сокращение времени учителя на проверку типовых заданий не менее чем на 30\%, уровень вовлечённости родителей (регулярность обращения к аналитике), а также успешная интеграция Системы в пилотных школах без необходимости глубокой перенастройки инфраструктуры.

% ===== 3 Характеристика объектов автоматизации
\section{Характеристика объектов автоматизации}
\subsection{Краткие сведения об объекте автоматизации}
Объектом автоматизации является образовательный процесс в общеобразовательных школах: взаимодействие учителя и ученика, проверка заданий, выдача рекомендаций, информирование родителей об успехах и трудностях. В настоящий момент большинство операций выполняется вручную, учителя используют разрозненные цифровые сервисы, а родители и администрация получают агрегированную информацию с задержкой. Отсутствие персонализированных маршрутов обучения ограничивает рост успеваемости и повышает нагрузку на педагога.

В пилотный контур входят три школы (городская, гимназия и школа с углублённым изучением предметов), где планируется развернуть рабочие места учителей-предметников, классных руководителей и администратора образования. Каждый пользователь взаимодействует с Системой через веб-интерфейс и мобильные уведомления, что позволяет сохранить привычные процессы и документы оборота.
\subsection{Сведения об условиях эксплуатации объекта автоматизация}
Эксплуатация Системы предполагается в стандартных условиях школьных компьютерных классов и на персональных устройствах учителей и учеников. Требуются доступ к сети Интернет со скоростью не ниже 20 Мбит/с, современные браузеры и устройства (ПК, ноутбуки, планшеты). В пилотных школах предусмотрено дежурное администрирование со стороны ИТ-службы для контроля авторизации и соблюдения политики безопасности. Специальных климатических или транспортных требований не предъявляется.

% ===== 4 Требования к системе
\section{Требования к системе}

\subsection{Требования к системе в целом}
\subsubsection{Требования к структуре и функционированию системы}
\paragraph{Перечень подсистем, их назначение и основные характеристики; требования к числу уровней иерархии и степени централизации системы} В состав Системы входят: \begin{enumerate}
  \item подсистема персонализированных AI-наставников с библиотекой преднастроенных ботов по предметам;
  \item подсистема кабинета учителя с журналом заданий, аналитикой и управлением группами;
  \item подсистема кабинета ученика с индивидуальными траекториями и адаптивным генератором практики;
  \item подсистема родительского контроля и уведомлений;
  \item интеграционно-аналитическая подсистема обмена данными с электронными журналами и школьными системами.
\end{enumerate}
Архитектура Системы облачная, многоарендная (SaaS) с единым центром управления доступами и сервисами аналитики.

\paragraph{Требования к способам и средствам связи для информационного обмена между компонентами системы} Компоненты взаимодействуют через защищённые REST и gRPC сервисы; для мгновенных уведомлений используется WebSocket. Все соединения шифруются по TLS 1.3, ключи доступа управляются через единый сервис авторизации.

\paragraph{Требования к характеристикам взаимосвязей создаваемой системы со смежными системами; совместимость (в т.ч. способы обмена информацией)} Система интегрируется с электронными журналами («Сетевой город», «МЭШ» и др.) посредством API или обмена CSV/JSON. Поддерживается единая авторизация по протоколам SAML 2.0/OAuth 2.0, экспорт отчётности предоставляется в форматах PDF и XLSX.

\paragraph{Требования к режимам функционирования системы} Система доступна 24/7, плановые работы проводятся не чаще двух раз в месяц в ночное время (до 2 часов). В учебные часы обеспечивается доступность не ниже 99\%. При авариях допустима временная деградация до режима чтения.

\paragraph{Требования по диагностированию системы} Для всех сервисов реализуется централизованное логирование и сбор метрик (Prometheus). Настраиваются оповещения по каналам e-mail и мессенджеру. Диагностические отчёты формируются ежемесячно.

\paragraph{Перспективы развития, модернизации системы} Планируется расширение библиотеки ботов, поддержка новых предметов и языков, подключение внешних AI-провайдеров, выпуск мобильных приложений. Масштабирование нагрузки обеспечивается контейнеризацией и оркестрацией.

\subsubsection{Требования к численности и квалификации персонала системы и режиму его работы}
\paragraph{Требования к численности персонала (пользователей) АС} Пилот рассчитан на 25 учителей, до 600 учеников и до 600 родителей. Со стороны школы выделяется минимум один администратор и два методиста. Исполнитель предоставляет менеджера проекта и инженера сопровождения.

\paragraph{Требования к квалификации персонала, порядку его подготовки и контролю знаний и навыков} Учителя проходят базовый тренинг (4 академических часа) и итоговое тестирование. Администраторы обучаются настройкам доступа и интеграциям (8 академических часов). Команда разработки включает 2 ML-инженеров, backend- и frontend-разработчика, продакт-менеджера, UX-дизайнера и QA-инженера.

\paragraph{Требуемый режим работы персонала АС} Учителя и ученики работают в рамках учебного расписания. Администратор системы контролирует доступ ежедневно не менее 1 часа. Поддержка Исполнителя доступна в режиме 5/2 с дежурствами по критическим инцидентам.

\subsubsection{Показатели назначения}
Ключевые метрики: снижение времени проверки типовых заданий на 30\%, наличие персональной траектории для 100\% учеников пилота, доля активных родителей не ниже 80\%, рост среднего балла по предметам пилотных классов на 5\%, оценка интерфейса учителями не ниже 4 из 5.

\subsubsection{Требования к надёжности}
Коэффициент готовности — 0,99 в учебные часы, время восстановления после критического отказа — до 2 часов. Резервные копии создаются ежедневно, хранятся 30 суток.

\subsubsection{Требования безопасности}
Обработка персональных данных соответствует 152-ФЗ и локальным актам школы. Реализуется ролевая модель доступа, двухфакторная аутентификация для учителей и администраторов, протоколирование действий за период не менее 180 дней.

\subsubsection{Требования к эргономике и технической эстетике}
Интерфейсы поддерживают завершение ключевых сценариев не более чем в три шага, адаптацию под ноутбуки и планшеты, выбор контрастных тем. Для пользователей доступны подсказки и единая навигация.

\subsubsection{Требования к транспортабельности для подвижных АС}
Система не относится к подвижным; специальных требований не предъявляется.

\subsubsection{Требования к эксплуатации, техническому обслуживанию, ремонту и хранению компонентов системы}
\paragraph{Условия и регламент (режим) эксплуатации} Эксплуатация ведётся в облачной инфраструктуре Исполнителя. Обновления выкладываются не чаще одного раза в две недели после тестирования на стенде. Регламент сопровождения описывается в эксплуатационной документации.
\paragraph{Предварительные требования к допустимым площадям для размещения персонала и ТС системы} Для пользователей требуются стандартные компьютерные классы или персональные устройства; специальных помещений не требуется. При локальном развертывании — отдельный шкаф с ИБП.
\paragraph{Требования по количеству, квалификации обслуживающего персонала и режимам его работы} Обслуживание обеспечивает инженер сопровождения и специалист по инфраструктуре Исполнителя. На стороне школы закреплён системный администратор.
\paragraph{Требования к составу, размещению и условиям хранения комплекта запасных изделий и приборов} Физические комплектующие не поставляются; резервирование обеспечивается облачными ресурсами.
\paragraph{Требования к регламенту} Регламент обслуживания включает ежедневный мониторинг, ежемесячные отчёты и процедуру выпуска обновлений с обязательным согласованием с Заказчиком.

\subsubsection{Требования к защите информации от несанкционированного доступа}
Данные шифруются при хранении (AES-256) и в канале. Доступ к административным функциям ограничивается IP-фильтрацией и многофакторной аутентификацией. Внедряется система обнаружения аномалий.

\subsubsection{Требования по сохранности информации при авариях}
Резервные копии и журналы транзакций хранятся в двух независимых регионах. Раз в месяц выполняется проверка восстановления. При аварии выполняется переход на резервную площадку.

\subsubsection{Требования к защите от влияния внешних воздействий}
Используются облачные средства защиты от DDoS, фильтрация трафика. На клиентских устройствах требуется актуальное антивирусное ПО.

\subsubsection{Требования к патентной чистоте}
Используемые AI-модели и контент лицензируются для образовательного применения. Перед внедрением проводится юридическая экспертиза материалов.

\subsubsection{Требования по стандартизации и унификации}
Система базируется на ГОСТ 34, ISO/IEC 27001, SCORM и xAPI. Форматы обмена и интерфейсы унифицированы во всех подсистемах.

\subsubsection{Дополнительные требования}
Необходимо реализовать управление подписками (бесплатный базовый доступ и платные расширения), поддержку русского и английского интерфейса, масштабируемость до 50 школ без модификации архитектуры.

\subsection{Требования к функциям (задачам), выполняемым системой}
\subsubsection{Подсистема персонализированных AI-наставников}
Генерирует объяснения и практические задания в диалоге с учеником, фиксирует прогресс и формирует рекомендации. Время отклика — до 3 секунд. Критическим отказом считается недоступность более 5 минут.

\subsubsection{Подсистема кабинета учителя}
Обеспечивает создание классов, планирование занятий, назначение заданий, мониторинг освоения тем и формирование отчётности. Формирование отчётов — до 30 секунд, доступ в PDF/XLSX.

\subsubsection{Подсистема родительского контроля и аналитики}
Отвечает за уведомления и витрину прогресса ученика. Синхронизация данных выполняется каждые 24 часа, критический отказ — недоставка уведомлений более 4 часов.

\subsection{Требования к видам обеспечения}
\subsubsection{Требования к математическому обеспечению}
Используются модели генеративного ИИ с ограничением на образовательные домены, модуль рекомендаций на основе градиентного бустинга/коллаборативной фильтрации. Требуется версионирование алгоритмов и настраиваемые параметры адаптивных тестов.

\subsubsection{Требования к информационному обеспечению}
\paragraph{Состав, структура и организация данных} Данные хранятся в PostgreSQL (структурированные сущности) и объектном хранилище (материалы). Сущности: пользователи, классы, траектории, задания, результаты, уведомления.
\paragraph{Информационный обмен и совместимость со смежными системами} Поддерживается импорт/экспорт расписаний и успеваемости в форматах CSV/JSON. Для массовых обменов используется очередь сообщений.
\paragraph{Использование классификаторов, применение СУБД, структура процесса сбора/обработки/передачи данных} Применяются классификаторы предметов по ФГОС, уровней сложности заданий, типов навыков. Сбор данных автоматизирован, передача — через защищённую шину.
\paragraph{Защита данных от разрушений, контроль/хранение/обновление/восстановление, придание юридической силы электронным документам} Контроль целостности обеспечивается проверкой checksum; электронные документы подписываются ЭП школы, история изменений хранится 3 года.

\subsubsection{Требования к лингвистическому обеспечению}
\paragraph{Требования к языкам программирования высокого уровня} Основные компоненты разрабатываются на Python и TypeScript, высоконагруженные сервисы допускают Go или Rust.
\paragraph{Требования к языкам взаимодействия пользователей и ТС системы} Интерфейсы доступны на русском и английском языках; предусмотрены готовые шаблоны промптов для учителей.
\paragraph{Требования к кодированию и декодированию данных} Используются UTF-8 и ISO 8601 для даты/времени, JSON Schema для обмена результатами тестов.
\paragraph{Требования к языкам ввода-вывода и манипулирования данными} Отчётность формируется на SQL и DSL аналитики, интеграции поддерживают GraphQL по требованию школ.
\paragraph{Требования к средствам описания предметной области (объекта автоматизации)} Предметная область описывается через онтологию навыков, совместима с SCORM/xAPI.
\paragraph{Требования к способам организации диалога} Диалоговый интерфейс реализован в веб-клиенте, поддерживает голосовой ввод и хранение истории диалогов.

\subsubsection{Требования к программному обеспечению}
ГОСТ 34.003–90. Программное обеспечение АС — совокупность программ на носителях данных и программных документов, предназначенная для отладки, функционирования и проверки работоспособности АС.
\paragraph{Требования к независимости программных средств от используемых СБТ и операционной среды} Компоненты развёртываются в контейнерах и не зависят от конкретной ОС; поддерживаются облачные платформы, удовлетворяющие требованиям импортозамещения.
\paragraph{Требования к качеству программных средств, способам его обеспечения и контроля} Внедряется процесс CI/CD, обязательное модульное и интеграционное тестирование, ревью кода. Покрытие тестами — не менее 70\%.
\paragraph{Требования по необходимости согласования разрабатываемых программных средств с фондом алгоритмов и программ} Алгоритмы и ПО регистрируются в реестре Заказчика; соблюдаются лицензии сторонних компонентов.

\subsubsection{Требования к техническому обеспечению}
ГОСТ 34.003–90. Техническое обеспечение АС — совокупность всех технических средств, используемых при функционировании АС.
\paragraph{Требования к видам технических средств} Используются облачные вычислительные мощности (GPU — для обучения моделей, CPU — для инференса). На стороне школ — ПК/планшеты с современными браузерами.
\paragraph{Требования к функциональным, конструктивным и эксплуатационным характеристикам средств технического обеспечения системы} Серверы должны поддерживать горизонтальное масштабирование, резервное питание и сетевые каналы. Клиентские устройства — поддержку WebRTC для голосового взаимодействия.

\subsubsection{Требования к метрологическому обеспечению}
\paragraph{Предварительный перечень измерительных каналов} Специальные физические измерения не требуются; контроль качества обеспечивается программно.
\paragraph{Требования к точности измерений параметров и/или к метрологическим характеристикам каналов} Погрешность прогнозов успеваемости — не более 10\%, точность классификации заданий — не ниже 85\%.
\paragraph{Требования к метрологической совместимости технических средств системы} Используются унифицированные шкалы оценивания (пятибалльная, процентная) с возможностью сопоставления с ФГОС.
\paragraph{Каналы, для которых оцениваются точностные характеристики; метрологическое обеспечение ТС и ПО; вид метрологической аттестации и порядок её выполнения} Проверка точности алгоритмов выполняется по методике, утверждённой методическим центром; переаттестация — ежегодно по результатам пилота.

\subsubsection{Требования к организационному обеспечению}
Организационное обеспечение АС — совокупность документов, устанавливающих организационную структуру, права и обязанности пользователей и эксплуатационного персонала АС, а также правовые нормы (реализуемые в организационном обеспечении).
\paragraph{Требования к структуре и функциям подразделений} На стороне школы формируется рабочая группа (заместитель директора, методист, системный администратор). Исполнитель создаёт управляющий комитет проекта.
\paragraph{Требования к организации функционирования системы и порядку взаимодействия персонала АС и персонала объекта автоматизации} Взаимодействие осуществляется через сервис-деск; регламент инцидентов устанавливает сроки реакции: P1 — 2 часа, P2 — 8 часов, P3 — 24 часа.
\paragraph{Требования к защите от ошибочных действий персонала системы} Реализуются мастер-настройки, подтверждения критических операций, журнал действий с возможностью отката.

\subsubsection{Требования к методическому обеспечению}
Методическое обеспечение включает руководство пользователя, карту внедрения на 6 месяцев, методики использования AI-наставников и шаблоны коммуникаций с родителями. Документы дополняются видеоматериалами и чек-листами качества уроков.

% ===== 5 Состав и содержание работ по созданию системы
\section{Состав и содержание работ по созданию системы}
Работы выполняются по стадиям ГОСТ 34.601–90 с учётом дорожной карты проекта:
\begin{enumerate}
  \item \textbf{Техническое задание} — разработка и согласование ТЗ, формирование требований (I квартал 2025 года).
  \item \textbf{Эскизный проект} — проектирование архитектуры, подготовка прототипов интерфейсов (II квартал 2025 года).
  \item \textbf{Технический проект} — разработка прототипа и завершение набора AI-ботов, подготовка эксплуатационной документации (III квартал 2025 года).
  \item \textbf{Рабочая документация и опытная эксплуатация} — пилотное внедрение в школах, корректировка по итогам обратной связи (IV квартал 2025 года — II квартал 2026 года).
  \item \textbf{Внедрение и сопровождение} — масштабирование на сеть школ, подготовка службы поддержки (III–IV кварталы 2025 года).
\end{enumerate}

\subsection{Перечень документов, предъявляемых по окончании соответствующих стадий и этапов работ}
По окончании стадий предоставляются: ТЗ, эскизный проект, технический проект, комплект эксплуатационной документации (руководство пользователя, программа и методика испытаний, план внедрения), отчёт об опытной эксплуатации и акт приёмки каждого этапа.

\subsection{Вид и порядок проведения экспертизы технической документации}
Экспертиза ТЗ проводится Заказчиком и методическим центром. Технический проект и программа испытаний проходят внутренний аудит Исполнителя и внешнюю экспертизу рабочей группы Заказчика. Результаты пилота рассматриваются приёмочной комиссией школы.

\subsection{Программа работ, направленных на обеспечение требуемого уровня надёжности разрабатываемой системы}
Предусмотрены нагрузочное тестирование, аудит безопасности, проверка резервного восстановления и контроль качества AI-моделей. Испытания выполняются перед каждым крупным релизом.

\subsection{Перечень работ по метрологическому обеспечению на всех стадиях создания системы}
Метрологические работы включают разработку методики оценки точности рекомендаций и ежегодную аттестацию алгоритмов по результатам пилота. Дополнительные физические измерения не требуются.

% ===== 6 Порядок контроля и приёмки системы
\section{Порядок контроля и приёмки системы}
\subsection{Виды, состав, объём и методы испытаний системы и её составных частей}
Испытания включают функциональное тестирование (100\% пользовательских сценариев), нагрузочное тестирование сервисов генерации заданий и отчётности, проверку корректности AI-рекомендаций на выборке учебных кейсов, а также опытную эксплуатацию в пилотных школах с учётом обратной связи пользователей.

\subsection{Общие требования к приёмке работ по стадиям}
По завершении каждой стадии оформляется акт приёмки. Прототип и отчёт об испытаниях рассматриваются рабочей группой Заказчика. Результаты пилота утверждаются приказом руководителя образовательной организации.

\subsection{Статус приёмочной комиссии}
Создаётся ведомственная приёмочная комиссия с участием представителей Заказчика, Исполнителя и методического центра. В состав включаются ответственные за ИТ-инфраструктуру и за методическое сопровождение.

% ===== 7 Требования к составу и содержанию работ по подготовке объекта автоматизации к вводу системы в действие
\section{Требования к составу и содержанию работ по подготовке объекта автоматизации к вводу системы в действие}
Подготовительные работы включают: инвентаризацию компьютерных классов, обновление ноутбуков и браузеров, обеспечение доступа к сети не ниже 20 Мбит/с, настройку единой системы авторизации, обучение персонала (учителей, администраторов, методистов), информирование родителей, сбор согласий на обработку данных и адаптацию локальных нормативных актов под работу с AI-наставниками.

% ===== 8 Требования к документированию
\section{Требования к документированию}
Документация по проекту формируется согласно ГОСТ 34.201 и включает комплекты: техническое задание, эскизный и технический проекты, эксплуатационные документы, материалы испытаний и отчёты по пилотам. Документы предоставляются в электронном виде (PDF/XLSX) и, при необходимости, на бумаге.

\begin{GOSTlongtable}{C{0.13\textwidth}|L{0.34\textwidth}|C{0.10\textwidth}|C{0.16\textwidth}|L{0.25\textwidth}}
\textbf{Стадия} & \textbf{Наименование} & \textbf{Код} & \textbf{Часть проекта} & \textbf{Комментарий} \\ \hline
ТЗ & Техническое задание на АС «Aivise» & ТЗ & ТЗ & Настоящий документ \\ \hline
ТП & Пояснительная записка к техническому проекту & П2 & ОР & Архитектура и подсистемы \\ \hline
ТП & Описание информационного обеспечения & П5 & ИО & Модель данных и интеграции \\ \hline
ТП & Описание программного обеспечения & ПА & ПО & AI-службы, веб-клиенты \\ \hline
ТП & Описание организационной структуры & ПВ & ОО & Роли учителей, учеников, родителей \\ \hline
РД & Программа и методика испытаний & ПМ & ОР & Функциональные и пилотные тесты \\ \hline
РД & Руководство пользователя (учитель, ученик, родитель) & И3 & ОО & Практические сценарии \\ \hline
РД & Руководство администратора & И3-А & ОО & Настройка доступов и интеграций \\ \hline
РД & Отчёт об опытной эксплуатации & ОЭ & ОР & Результаты пилота \\ \hline
РД & Паспорт системы & ПС & ОР & Основные технические параметры \\ \hline
\end{GOSTlongtable}

% ===== 9 Источники разработки
\section{Источники разработки}
При разработке ТЗ и проектных решений использовались: \begin{enumerate}
  \item Концепция платформы Aivise (версия 1.1 от 20.09.2025).
  \item Аналитический отчёт «Практики внедрения AI в школах РФ» (сентябрь 2025 года).
  \item Результаты интервью с учителями, родителями и администрацией пилотных школ (октябрь 2025 года).
  \item Материалы конкурентного анализа (Kahoot, ChatGPT, электронные журналы) и маркетинговая стратегия вывода на рынок (ноябрь 2025 года).
\end{enumerate}

% ---- Подписи / согласования (страницы)
\SignatureTables

\end{document}
