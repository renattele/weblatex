% ==== ГОСТ 34: ПРЕАМБУЛА И МАКРОСЫ (без \documentclass и без \begin{document}) ====

% Кодировки и язык (pdfLaTeX; для XeLaTeX используйте fontspec/polyglossia)
\usepackage[T2A]{fontenc}
\usepackage[utf8]{inputenc}
\usepackage[russian]{babel}
\hyphenation{долж-ность ис-пол-ни-те-ля под-пись функ-ци-о-наль-ные ста-ди-я ком-мен-та-рий про-гра-мма ме-то-ди-ка ис-пы-та-ний}
\uchyph=1 % разрешаем переносы слов с прописными буквами

% Поля, абзацы, межстрочный
\usepackage[left=30mm,right=10mm,top=20mm,bottom=20mm]{geometry}
\usepackage{indentfirst}
\usepackage{setspace}
\onehalfspacing
\setlength{\parindent}{1.25cm}
\setlength{\parskip}{0pt}

% Таблицы, рисунки, подписи
\usepackage{graphicx}
\usepackage{array}
\usepackage{ragged2e} % гибкие выравнивания с переносами в таблицах
\usepackage{longtable}
\usepackage{caption}
\DeclareCaptionLabelSeparator{emdash}{\textemdash\ }
\captionsetup{labelsep=emdash, justification=raggedright, singlelinecheck=false}
\renewcommand{\figurename}{Рисунок}
\renewcommand{\tablename}{Таблица}

% Глобальный стиль таблиц: рамка/сетка, заметные линии
\setlength{\arrayrulewidth}{0.8pt}
\setlength{\tabcolsep}{6pt}
\renewcommand{\arraystretch}{1.2}

% Обёртки-окружения для таблиц в стиле ГОСТ
\newenvironment{GOSTtabular}[1]{%
  \begin{tabular}{|#1|}\hline
}{%
  \end{tabular}
}
\newenvironment{GOSTlongtable}[1]{%
  \setlength{\LTleft}{0pt}\setlength{\LTright}{0pt}% выравнивание по центру
  \centering
  \begin{longtable}{|#1|}
  \hline
}{%
  \end{longtable}
}

% Полезные p-колонки
\newcolumntype{L}[1]{>{\RaggedRight\arraybackslash}p{\dimexpr#1-2\tabcolsep\relax}}
\newcolumntype{C}[1]{>{\Centering\arraybackslash}p{\dimexpr#1-2\tabcolsep\relax}}
\newcolumntype{R}[1]{>{\RaggedLeft\arraybackslash}p{\dimexpr#1-2\tabcolsep\relax}}

% Списки и глубина нумерации (до 1.1.1.1)
\usepackage{enumitem}
\setlist{nosep,topsep=0pt,partopsep=0pt}
\setcounter{secnumdepth}{4}
\usepackage{titlesec}
\titleformat{\section}{\normalfont\bfseries\large}{\thesection}{1em}{}
\titleformat{\subsection}{\normalfont\bfseries\normalsize}{\thesubsection}{1em}{}
\titleformat{\subsubsection}{\normalfont\bfseries\normalsize}{\thesubsubsection}{1em}{}
\titleformat{\paragraph}{\normalfont\bfseries\normalsize}{\theparagraph}{1em}{}

% Ссылки (ч/б без цветов)
\usepackage[unicode]{hyperref}
\hypersetup{colorlinks=true, linkcolor=black, urlcolor=black, citecolor=black}

% ===== Макросы метаданных титула
\newcommand{\OrgName}{ООО «Aivise Technologies»}
\newcommand{\SystemName}{Автоматизированная образовательная система персонализированного сопровождения «Aivise»}
\newcommand{\DocType}{ТЕХНИЧЕСКОЕ ЗАДАНИЕ} % / ЧАСТНОЕ ТЕХНИЧЕСКОЕ ЗАДАНИЕ
\newcommand{\TopicCode}{AIVISE-EDU/2025-ТЗ}
\newcommand{\City}{Казань}
\newcommand{\Year}{2025}

% ===== Титульный лист (вызывается один раз в основном файле)
\newcommand{\MakeTitlePage}{%
\begin{titlepage}
\thispagestyle{empty}
\begin{center}
{\large \OrgName\par}
\vspace{25mm}
{\Large \bfseries \SystemName\par}
\vspace{10mm}
{\LARGE \bfseries \DocType\par}
\vspace{5mm}
{\large Шифр темы: \TopicCode\par}
\vfill
{\City\par}
{\Year\par}
\end{center}
\end{titlepage}
}

% ===== Перечень сокращений — таблица (GOSTlongtable)
\newcommand{\AbbrevTable}{%
\section*{ПЕРЕЧЕНЬ СОКРАЩЕНИЙ И УСЛОВНЫХ НАИМЕНОВАНИЙ}
\addcontentsline{toc}{section}{ПЕРЕЧЕНЬ СОКРАЩЕНИЙ И УСЛОВНЫХ НАИМЕНОВАНИЙ}
\begin{GOSTlongtable}{L{0.30\textwidth}|L{0.62\textwidth}}
\textbf{Термин, сокращение} & \textbf{Определение} \\ \hline
АС & Автоматизированная система \\ \hline
АСУ & Автоматизированная система управления \\ \hline
САПР & Система автоматизированного проектирования \\ \hline
КТС & Комплекс технических средств \\ \hline
% Добавляйте строки ниже:
% АСНИ & Автоматизированная система научных исследований \\ \hline
\end{GOSTlongtable}
\clearpage
}

% ===== Пример обычной таблицы (унифицированный стиль ГОСТ)
\newcommand{\DemoTable}{%
\begin{table}[h]
\centering
\caption{Наименование таблицы}
\begin{GOSTtabular}{L{0.3\textwidth}|L{0.65\textwidth}}
\textbf{Заголовок столбца 1} & \textbf{Заголовок столбца 2} \\ \hline
Ячейка 1 & Ячейка 2 \\ \hline
Ячейка 3 & Ячейка 4 \\ \hline
\end{GOSTtabular}
\end{table}
}

% ===== Таблицы "СОСТАВИЛИ / СОГЛАСОВАНО" в стиле эталона
\newcommand{\SignatureTables}{%
  \clearpage
  \thispagestyle{plain}%
  \begin{center}
    {\small \TopicCode. Техническое задание / Частное техническое задание}
  \end{center}

  \vspace{2mm}
  \begin{center}\textbf{\MakeUppercase{СОСТАВИЛИ}}\end{center}
  \vspace{1mm}

  \begin{center}
  \begin{minipage}{\textwidth}
    \setlength{\tabcolsep}{3pt}% ужимаем отступы внутри таблицы подписей
    \begin{GOSTtabular}{L{45mm}|L{32mm}|L{37mm}|C{24mm}|C{18mm}}
      \textbf{Наименование организации, предприятия} &
      \textbf{Должность исполнителя} &
      \textbf{Фамилия, имя, отчество} &
      \textbf{Подпись} &
      \textbf{Дата} \\ \hline
      & & & & \\ \hline
      & & & & \\ \hline
      & & & & \\ \hline
      & & & & \\ \hline
    \end{GOSTtabular}
  \end{minipage}
  \end{center}

  \vspace{6mm}
  \begin{center}\textbf{\MakeUppercase{СОГЛАСОВАНО}}\end{center}
  \vspace{1mm}

  \begin{center}
  \begin{minipage}{\textwidth}
    \setlength{\tabcolsep}{3pt}% ужимаем отступы внутри таблицы подписей
    \begin{GOSTtabular}{L{45mm}|L{32mm}|L{37mm}|C{24mm}|C{18mm}}
      \textbf{Наименование организации, предприятия} &
      \textbf{Должность исполнителя} &
      \textbf{Фамилия, имя, отчество} &
      \textbf{Подпись} &
      \textbf{Дата} \\ \hline
      & & & & \\ \hline
      & & & & \\ \hline
      & & & & \\ \hline
      & & & & \\ \hline
    \end{GOSTtabular}
  \end{minipage}
  \end{center}

  \clearpage
}
